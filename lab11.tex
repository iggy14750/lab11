
\documentclass[titlepage]{article}
\usepackage[a4paper, total={6in, 10in}]{geometry}
\usepackage{graphicx}
\author{Isaac B Goss\\Performed March 7 - 21}
\title{Lab 11: Register Files}
\begin{document}
	\begin{titlepage}
	    \maketitle
	\end{titlepage}

    \section{Purpose}

%        With this assignment, we are to further our digital design skills.
        Any remotely useful computation machine involves memory, and any useful engineer knows how to use it.
        To this end, a particular specification, one of many ways to arrange our given puzzle pieces, was presented.
        This circuit consisted of two counters being fed into the read and write input of a register file, four switches fed into its write data input, and two seven segment displays, fed through decoders, one from the read data output of our register file, and the other directly from our switches.
%        The following lab demonstrates my own use of a memory device, a 4x4 register file.
        
        So what does it do?
        In short, this design displays what is about to be written to memory, and what is currently being read from memory.
        What makes memory useful is that these two activities are distinct.  
        The operation of this circuit is intended to make this very clear.
        In the following, I plan to demonstrate how my memory device fulfills this characteristic.
        
    \section{Procedure}
    
        \begin{figure}[h]
        	\centering
        	\includegraphics[width=0.8\textwidth]{design1}
        	\caption{Design with Seven-Segments}
        \end{figure}
        
        As instructed, this specification was first tested in software, using Quartus II.
        In Figure 1, you can see our design, which consists of two 2-bit counters which choose the read and write locations of our register file.
        You see our 4-bit input, bussed to both the write data input of our register file, and directly into a decoder.
        This decoder then feeds into a seven segment display in order to see what we are writing to memory.
        Likewise, the read data output of the register file is fed into another decoder, and into yet another seven-segment display, so that it is possible to see what is \textit{read} from memory.
        
        For simulation purposes, however, the seven-segments were removed, as you can see in the next figure.  Their inputs were bussed directly into two output terminals for easier reading. 
        The details of this simulation will be discussed in the \textit{Results} section.
        
        \begin{figure}[h]
        	\centering
        	\includegraphics[width=0.8\textwidth]{design2}
        	\caption{Design for Simulation}
        \end{figure}
        
        Then came bread-board implementation.  The picture at the end of this report illustrates how our build was arranged.
        On the left, you see two SPDT switches, each acting as the UP inputs to the counters (middle, upper) similarly described in the Quartus design.
        Looking down, we have an 8-bit block of DIP toggles, which act as the 4-bit input from before.
        This, unsurprisingly, is fed both into the write data input of our 4x4 reg file (middle board), and into another decoder (middle, lower).
        The read data output of the reg file is fed into a decoder (middle, second from bottom), and both these decoders output to their respective seven-segments (right).
        Top right you see one more switch, which is de-bounced and fed into write enable in our reg file.
        
        The last step was measuring our circuit's performance with Logic Analyzer.  As before, this test will be described in \textit{Results}.  
        Our testing method is worth mentioning, however.
        Our test was a 30-second capture, wherein we wrote four numbers to memory, and then read them out.  
        This was a four-handed procedure, where I managed write and read addresses and write enable, and my partner changed the number to be written to memory with each ``cycle''.
        
        
    \section{Results}
    
        \begin{figure}[h]
        	\centering
        	\includegraphics[width=\textwidth]{simulation}
        	\caption{Quartus Simulation Data}
        \end{figure}
    
        Above, one can see our Quartus simulation.  The top bit, ``Address\_in'', represents the UP input to our write counter.
        That is, with each rising edge, the next address was being written in our reg file.
        The next, similarly, represent a change in the read location from our reg file.
        The following group captures the input bits, which initialize at 8, hold 8 through one clock cycle (Address\_in cycle), and then change to 4 and then 2 with each cycle.
        Following this, the output of the decoder is presented, named ``LED\_IN''.  
        This gives us a one-cold representation of the input bus, the bits immediately above.  
        This representation, of course, is fed directly into a seven-segment display.
        
        The next section, labeled ``LED\_OUT'', 
        which is a similar representation of the data currently being read from memory, ready for seven-segment display.
        Finally, we have ``REGO'', short for ``register out''.  This is the base-2 representation of the data being read from memory.
        Notice that these two track together, as they are simply different representations of the same data.
        Also, they remain unchanged until later in the experiment.  This is true because ``Address\_out'' was static until this time.  That is, the address read from memory remained unchanged, so although more data was written in other areas of memory, the data read remained the same.
        
        \begin{figure}[h]
        	\centering
        	\includegraphics[width=\textwidth]{testing}
        	\caption{Logic Analyzer Data}
        \end{figure}
        
        Figure 4 shows the Logic Analyzer reading of our physical build.
        The reader may notice the extra time at the beginning and end of the timing diagram.
        This uneventful space exists due to our capture mechanism, described in \textit{Procedure}.
        The following table describes the relationship between the bit groups in these two figures, and their function.
        
        \begin{center}
        \begin{tabular}{ll | l}
        	
            Logic Analyzer & Quartus & Function \\
            \hline
            W & Address\_in & Write Address \\
            IN & IN & Input Data \\
            LED\_IN & LED\_IN & Input Data in One-Cold \\
            R & Address\_out & Read Address \\
            OUT & REGO & Read Data \\
            LED\_OUT & LED\_OUT & Read Data in One-Cold \\
            
        \end{tabular}
        \end{center}
        These two figures say tell the same story; that is, we can change what and where we are \textit{writing} to memory without changing what's \textit{read} from memory -- provided that these happen in different addresses.
        More to the point, that writing data can happen any time before reading that data back, the fundamental characteristic of memory.
        
    \section{Conclusions}
        
        In short, implementing a design with a register file was a relatively simple process, largely a matter of putting together the puzzle pieces earlier discovered.
        Our design performed as expected, both in soft and hard formats.
        This, of course, is plain in our data.
        Therein, as discussed in \textit{Results}, we see the same behavior, namely, that data is stored in our register file independent of what is being read.
        It is clear from this exercise the important considerations of memory implementation.
        
    \section{References}
        Aaron Marko was my lab partner.
        
%    \begin{figure}[h]
%    	\centering
%    	\includegraphics[width=\textwidth]{breadboard}
%    \end{figure}
%    
\end{document}









